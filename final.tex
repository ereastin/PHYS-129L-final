\documentclass[12pt]{article}
\usepackage{fancyhdr}
\usepackage{graphicx}
\usepackage{color}
\usepackage{hyperref}

\usepackage[hmargin=90bp,tmargin=108bp,bmargin=72bp,
            headheight=15bp,footskip=40bp]{geometry}

\newcommand\thisis{Solar Event Imaging Program}
\newcommand\theauthor{Evan~Eastin}

\newcommand\sfb{\sffamily\bfseries}

\newcommand\red[1]{\textcolor{red}{\sffamily\bfseries #1}}

\fancypagestyle{firstpg}
   {
   \fancyhf{}%
   \cfoot{\sffamily\thepage}%
   \renewcommand\headrulewidth{0bp}
   }

\pagestyle{fancy}
\lhead{\sffamily \thisis}
\chead{}
\rhead{\sffamily \theauthor}

\lfoot{}
\cfoot{\sffamily\thepage}
\rfoot{}

\begin{document}
\thispagestyle{firstpg}

\noindent
{\sffamily\bfseries\huge \thisis}\\

\noindent
{\large\sffamily \theauthor}

\vspace*{20bp}

\noindent
Part I: Necessary Software Installations
This program requries installation of SunPy, AstroPy, and Pillow. All three pieces of software can be installed by running the command pip install (sunpy, astropy, Pillow) from the terminal command line. Pillow by default also requires the installation of 2 external libraries, libjpeg and zlib. These can be installed by running apt-get install zlib libjpeg from the terminal command line.

Part II: Project Description
This program is intended to retrieve and display images of the sun and specific solar events. Within the program, the user is instructed to enter a time interval to search, what specific solar event they would like to see, what observatory/telescope/detector to use via the source ID, and what colormap to use for plotting. The program, making use of sunpy, astropy, and Pillow modules, queries the Heliophysics Event Knowledgebase with the entered time interval and event type and returns a list of recorded events within the time interval and properties specific to each event. The program then parses the information returned to retrieve the helioprojective cartesian x and y coordinates (in arcseconds), as well as the start and end times of the event closest to the start of the user requested time interval search. The start time of the event is reformatted to match the required date format when querying Helioviewer. The Helioviewer client request is then set up to retrieve and image at the requested date. Lists containing the numerical detector IDs and colormap options are printed for the user to select from. Once the desired source and colormap are selected, the Helioviewer query returns a PNG image of the sun from the desired date. Using Pillow, the PNG image is converted to JPG format, then split into RGB components. The RGB component arrays are translated to FITS data types and recompiled into a single FITS file for the sunpy Maps module to use. This file is plotted in the user requested colormap centered at the center of solar disk. The program then defines a region of interest centeerd on the coordinates of the event, scaled to match the scale of the plotted image, returned by the HEK query. This region of interest is mapped to a subplot that shows a zoomed in image of the region of interest with a size of 120 by 120 arcseconds in the same colormap as the original plot. A matplotlib module is used to clip the brightest pixels out for better viewing, and the images are plotted. The fits and jpg files created within the program are removed once the program has finished running for continued use.

Part III: Results
The program should run without issue, and can return a number of different plots and event types. Due to this version of sunpy only able to save plots as fits files I have included a screenshot of a plot a CME recorded on 2011-06-07 at 06:33:07 UTC plotted in the sdoaia131 colormap in the project directory (Latex was unable to read this png file). This result can be returned by entering a time interval 2011/06/07 to 2011/06/08, selecting CME, and selecting the AIA 193 data source with ID 11. The program can display a variety of colormaps and events that are worth looking into.

\end{document}

